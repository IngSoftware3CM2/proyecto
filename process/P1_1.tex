%========================================================
%Eliminar Personal
%========================================================

%========================================================
% Descripción general del proceso
%-----------------------------------------------
\begin{Proceso}{P1.1}{Eliminar Personal} {
  
    %-------------------------------------------
    %Resumen
    \bigskip El proceso inicia cuando el \cdtRef{Actor:Departamento de Capital Humano}{Departamento de Capital Humano} solicita la eliminación de un elemento, después dicha solicitud debe ser enviada y procesada por la biblioteca central del politécnico. Habiendo sido aceptada la solicitud, la biblioteca central genera un reporte que se regresa a la biblioteca local, con el cual elimina al personal por medio de una petición al sistema en la cual se envía el usuario y contraseña del elemento a eliminar.
  
    
    %-------------------------------------------
    %Diagrama del proceso

    \noindent La Figura \cdtRefImg{P1.1}{Eliminar Personal} muestra las actividades que se realizan para llevar a cabo el proceso descrito anteriormente.

    \Pfig[0.95]{./process/images/Proceso-Dias-Economicos.png}{P1.1}{Eliminar Personal}}{P1.1:Eliminar Personal}

    %-------------------------------------------
    %Elementos del proceso

    \UCitem{Usuario} { %Actores
        \cdtRef{Actor:Jefe de Biblioteca}{Jefe de Biblioteca}.
        \cdtRef{Actor:Sistema}{Sistema}.
    }

    \UCitem{Objetivo} { %Objetivo
        Eliminar personal del sistema.
    }

    \UCitem{Insumos de entrada}{ %Insumos de entrada
  	    \begin{UClist}
  		    \UCli Contraseña del \cdtRef{Actor:Departamento de Capital Humano}{Jefe de Biblioteca}.
  		    \UCli Contraseña del \cdtRef{Actor:Jefe de Biblioteca}{Jefe de Biblioteca}.
        \end{UClist}
    }
  
    \UCitem{Proveedores}{ %Proveedores
        Biblioteca central.
    }

    \UCitem{Productos de salida}{ %Productos de salida
        \begin{UClist}
            \UCli Notificación \cdtIdRef{MSJ1.5}{Eliminacion exitosa}.
        \end{UClist}
    }

    \UCitem{Cliente}{ %Cliente
        \cdtRef{Actor:Jefe de Biblioteca}{Jefe de Biblioteca}
    }

    \UCitem{Mecanismo de medición}{ %Mecanismo de medición
        \begin{UClist}
            \UCli Respuesta inmediata
        \end{UClist}
    }
    
    \UCitem{Interrelación con otros procesos} { %Interrelación con otros procesos
        \cdtIdRef{P1.2}{Modificar Personal}
    }

\end{Proceso}

%========================================================
%Descripción de tareas
%-----------------------------------------------
\begin{PDescripcion}
  %Actor: Jefe de Bibliotecca
    \Ppaso Jefe de Biblioteca
        \begin{enumerate}
        %Tarea a
            \Ppaso[\itarea] \cdtLabelTask{T1-P1.3:Jefe de Biblioteca}{Solicitar eliminar empleado.}El \cdtRef{Actor:Departamento de Capital Humano}{Jefe de Biblioteca} solicita a biblioteca central por medio de un documento escrito, la aprobación para la eliminación de un elemento del personal de biblioteca. 
            \Ppaso[\itarea] \cdtLabelTask{T2-P1.3:Jefe de Biblioteca}{Recibe reporte de eliminacion.}El jefe de la biblioteca recibe un reporte con los datos del empleado y la autorización o rechazo de la eliminación del elemento.
            \Ppaso[\itarea] \cdtLabelTask{T3-P1.3:Jefe de Biblioteca}{Recibe reporte de eliminacion.}Ingresa el usuario y contraseña del empleado a eliminar dentro del sistema y acepta la eliminación.
        \end{enumerate}
    
    %Actor: biblioteca central
    \Ppaso Biblioteca Central
        \begin{enumerate}
            %Tarea a
            \Ppaso[\itarea] \cdtLabelTask{T1-P1.3:Biblioteca Central}{Recibe Solicitud de eliminacion.}Recibe la petición para la eliminación del empleado y determina si es aprobada o rechazada según las políticas internas de dicha biblioteca.
            \Ppaso[\itarea] \cdtLabelTask{T2-P1.3:Biblioteca Central}{Genera reporte de eliminacion.}Se genera un reporte completo con los datos del empleado y los motivos de eliminación, así como las fechas exactas en las que termina su participación en su labor. El reporte se envía de vuelta al jefe.
        \end{enumerate}    
    
    %Actor: Sistema
    \Ppaso Sistema
        \begin{enumerate}
         %Tarea a
            \Ppaso[\itarea] \cdtLabelTask{T1-P1.3:Sistema}{Elimina empleado.}Elimina al empleado del sistema y envía los datos de la eliminación al jefe de biblioteca.  
        \end{enumerate} 
\end{PDescripcion}
