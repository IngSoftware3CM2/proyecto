% Proceso: Licencia por cuidados familiares
%========================================================
% Descripción general: proceso para justificar incidencia por motivo de cuidados familiares
%========================================================


\begin{Proceso}{P1.3}{Cuidados familiares} {
  
    %-------------------------------------------
    %Resumen
    
    \smallskip
    \bigskip El proceso inicia cuando el \cdtRef{Actor: Personal Docente}{Personal Docente} o \cdtRef{Actor: Personal PAAE}{Personal de Apoyo y Asistencia a la Educacion} requiere justificar una incidencia por el motivo de cuidados familiares por algun accidente que amerite hospitalizacion o enfermedad del cónyuge, hijos, padres y dependientes económicos. Para esto debe solicitar a \cdtRef{Actor: ISSSTE}{Instituto de Seguridad y Servicios Sociales de los Trabajadores del Estado} una constancia médica, recomendando cuidados especiales y además acreditar la dependencia económica del familiar ante el  \cdtRef{Actor: ISSSTE}{ISSSTE} y enseguida entregarla en el área de \cdtRef{Actor: Departamento de Capital Humano}{Departamento de Capital Humano} para que pueda ser firmada de recibido, para esperar la respuesta de aceptacion por parte de  \cdtRef{Actor: Departamento de Capital Humano} o en su defecto ser rechasada.
    

    %-------------------------------------------
    %Diagrama del proceso
    
    \smallskip
    \noindent La Figura \cdtRefImg{P1.3}{Cuidados familiares} señala las actividades que se realizaran para realizar la justificacion de la incidencia por motivo de cuidados familiares.
    \\\\\\\\\\\\\\\\\\\
    \Pfig[0.90]{./process/images/procesoCuidadosFamiliares.png}{P1.3}{Cuidados familiares}}{P1.3:Cuidados familiares}
    

    %-------------------------------------------
    %Elementos del proceso

    \UCitem{Usuario} { %Actores
        \cdtRef{Actor: Departamento de Capital Humano}{Departamento de Capital Humano}.
        \cdtRef{Actor: Personal Docente}{Personal Docente}.
        \cdtRef{Actor: ISSSTE}{Instituto de Seguridad y Servicios Sociales de los Trabajadores del Estado}.
    }

    \UCitem{Objetivo} { %Objetivo
        Justificar licencia por cuidados familiares.
    }

    \UCitem{Insumos de entrada}{ %Insumos de entrada
  	    \begin{UClist}
  		    \UCli Constancia medica emitida por \cdtRef{Actor: ISSSTE}{Instituto de Seguridad y Servicios Sociales de los Trabajadores del Estado}.
  		    \UCli Acreditar la dependencia médica de los familiares ante el \cdtRef{Actor: ISSSTE}{Instituto de Seguridad y Servicios Sociales de los Trabajadores del Estado}.
  		\end{UClist}
    }
  
    \UCitem{Proveedores}{ %Proveedores
        \cdtRef{Actor: ISSSTE}{Instituto de Seguridad y Servicios Sociales de los Trabajadores del Estado}.
    }

    \UCitem{Productos de salida}{ %Productos de salida
        \begin{UClist}
            \UCli Confirmacion de registro de incidencia exitoso.
        \end{UClist}
    }

    \UCitem{Cliente}{ %Cliente
        \cdtRef{Actor: Departamento de Capital Humano}{Departamento de Capital Humano}.
    }
    
    \UCitem{Interrelación con otros procesos} { %Interrelación con otros procesos
        \cdtIdRef{P1.1}{Macroproceso de incidencias}
    }

\end{Proceso}

%-----------------------------------------------
%Descripcion de tareas

\begin{PDescripcion}
  %Actor: Personal Docente
    \Ppaso Personal Docente
        \begin{enumerate}
        %Tarea a
            
            \Ppaso[\itarea] \cdtLabelTask{T1-P1.3:Personal Docente}{Solicita su constancia médica al \cdtRef{Actor: ISSSTE}{ISSSTE}.} El \cdtRef{Actor: Personal Docente}{Personal Docente} debe entregar la constancia al cdtRef{Actor: Departamento de Capital Humano}{Departamento de Capital Humano} ya que si no esta no procedera y se tomara en cuenta como falta sin goce de sueldo.
            \Ppaso[\itarea] \cdtLabelTask{T2-P1.3:Personal Docente}{Esperara a que reciba la confirmacion de registro exitoso de la incidencia.}
        \end{enumerate}
    
   %Actor: Departamento de Capital Humano 
    \Ppaso Departamento de Capital Humano
        \begin{enumerate}
            %Tarea 
            \Ppaso[\itarea] \cdtLabelTask{T1-P1.3: Departamento de Capital Humano}{Recibe constancia médica.} Se recibe la constancia médica por parte del  \cdtRef{Actor: Personal Docente}{Personal Docente} o \cdtRef{Actor: Personal PAAE}{Personal de Apoyo y Asistencia a la Educacion} y esta se registrara en el sistema de un tercero para que se contemple la justificacion.
            \Ppaso[\itarea] \cdtLabelTask{T2-P1.3: Departamento de Capital Humano}{Verifica registro exitoso y envia mensaje de confirmacion.} Al registrar la constancia y al ser todo exitoso se le notificara a para que contemple el que se a realizado la justificacion por su inasistencia.
        \end{enumerate} 
    
\end{PDescripcion}
