%=========================================================
\label{cap:glosario}


\hyperlink{capitalHumano}{Este es un ejemplo}
%%%%%%%%%%%%% DEFINICIONES %%%%%%%%%%%%%
\section{Definiciones}
    \begin{itemize} 
        %%%%% A %%%%%
        \item \textbf{Alcance:}  Marco hacia donde enfocar el desarrollo del proyecto.
        \item \textbf{Análisis:} Es la parte del proceso de desarrollo de software cuyo propósito principal es realizar un modelo del dominio del problema. El análisis hace foco en qué hacer, el diseño hace foco en cómo hacerlo.
        \item \textbf{Artefacto:} Es una información que es usada o producida mediante un proceso de desarrollo de software. Un artefacto puede ser un modelo, una descripción o software.
        %%%%% D %%%%%
        \item \textbf{Disposición Final:} La acción de depositar o confinar permanentemente residuos sólidos en sitios o instalaciones cuyas características  prevean afectaciones a la salud de la población y a los ecosistemas y sus elementos.
        %%%%% E %%%%%
        \item \textbf{Especificación:} Es un informe de acuerdo entre el implementador y el usuario.
        \item \textbf{Especificación de requerimientos:} Es aquella que establece un acuerdo entre el usuario y el desarrollador del sistema.
        \item \textbf{Estaciones de Transferencia:} Las instalaciones para el trasbordo de los residuos sólidos de los vehículos de recolección a los vehículos de transferencia.
        %%%%% G %%%%%
        \item \textbf{Generación:} La acción de producir residuos sólidos a través de procesos productivos o de consumo.
        \item \textbf{Gestión del riesgo:} Función ejecutiva que se encarga de mantener bajo control los daños y perjuicios que una organización puede encontrar en sus actividades.
        %%%%% I %%%%%
        \item \textbf{Ingeniería del Sotfware:} Es una disciplina para el desarrollo de software de alta calidad para sistemas basados en computadora.
        %%%%% M %%%%%
        \item \textbf{Modelo:} Es una abstracción semánticamente consistente de un sistema.
        %%%%% P %%%%%
        \item \textbf{Plan de contingencia:} Manejo de planes alternativos ante falla en el plan original.
        \item \textbf{Plan de manejo:} El Instrumento cuyo objetivo es minimizar la generación y maximizar la valorización de residuos sólidos urbanos y residuos de manejo especial, bajo criterios de eficiencia ambiental, tecnológica, económica y social, diseñado bajo los principios de responsabilidad compartida y manejo integral, que considera el conjunto de acciones, procedimientos y medios viables e involucra a productores, importadores, exportadores, distribuidores, comerciantes, consumidores, usuarios de subproductos y grandes generadores de residuos, según corresponda, así como a los tres niveles de gobierno.
        %%%%% R %%%%%
        \item \textbf{Requerimiento:} Es una característica, propiedad o comportamiento deseado para un sistema.
        \item \textbf{Riesgo:} Medida de la probabilidad y gravedad de sufrir efectos adversos inherentes al desarrollo de software que no cumpla con sus requerimientos.
        %%%%% S %%%%%
         \item \textbf{Stakeholder:} Son todas aquellas personas u organizaciones que afectan o son afectadas por el proyecto, ya sea de forma positiva o negativa.

    \end{itemize}
    
%%%%%%%%%%%%% ACRÓNIMOS %%%%%%%%%%%%%
\section{Acrónimos}
    \begin{itemize} 
    	\item \textbf{SIRS:} Sistema de Información de Residuos Sólidos
     	\item \textbf{SEDEMA:} Secretaria del Medio Ambiente
     	\item \textbf{SOS:} Secretaría de Obras y Servicios.
     	\item \textbf{DPEP: } Dirección de Planeación y Evaluación de Proyectos.
     	\item \textbf{DGRVA:} Dirección General de Regulación y Vigilancia Ambiental.
        \item \textbf{DGSU:} Dirección General de Servicios Urbanos.
        \item \textbf{INEGI:} Instituto Nacional de Estadística, Geografía e Informática.
        \item \textbf{JUD:} Jefe de Unidad Departamental.
        \item \textbf{LAUDF: }Licencia Ambiental Única para el Distrito Federal.
        \item \textbf{DCH: } Dirección de Capital Humano
        \item \textbf{SDCH: } Sistema de Dirección de Capital Humano
    \end{itemize}

